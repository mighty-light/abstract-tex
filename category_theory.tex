\documentclass{scrbook}

\usepackage{amsmath}
\usepackage{amssymb}
\usepackage{amsthm}
\usepackage{xcolor}
\usepackage{hyperref}
\usepackage{datetime}

\newdateformat{monthyear}{
  \monthname[\THEMONTH] \THEYEAR
}

\theoremstyle{definition}\newtheorem{defn}{Definition}
\theoremstyle{definition}\newtheorem*{rmk}{Remark}

\newcommand{\Chapter}[1]{\chapter*{#1}\addcontentsline{toc}{chapter}{#1}}
\newcommand{\hi}[1]{{\color{blue}\textbf{#1}}}

\newcommand{\Z}{\mathbf{Z}}
\newcommand{\Q}{\mathbf{Q}}
\newcommand{\R}{\mathbf{R}}

\DeclareMathOperator{\Hom}{Hom}
\DeclareMathOperator{\Id}{Id}

\author{\textsc{Bhavya Tiwari}\thanks{Roll number \textsc{24B0913}}}
\title{Category Theory \\[5mm]
\Large Summer of Science 2025 Project}

\begin{document}
\maketitle
\tableofcontents

\Chapter{Preface}

This report is a collection of things I have learnt and problems that I have
solved during my Summer of Science 2025 project on \emph{Category Theory} under
Aryaman Maithani.
\\[2mm]
If some advanced example from \emph{Category Theory in Context} took some time
to process, I add a corresponding chapter in the ``Crash Courses'' part so that
it is easier to understand. Nevertheless, it is supposed to be quite concise
(and therefore may prove to be insufficient). In my defence,
we can't keep reading a new book to understand an example in a category theory
book; the program is called Summer of Science and not Sages of Science.
\\[2mm]
More Details to be added later.

\bigskip\noindent
Bhavya Tiwari, IIT Bombay \\
\monthyear\today

\part{Introduction to Category Theory}

\chapter{Introduction to Categories}

\section{What is a Category?}

\begin{defn}[Category]
  A \emph{category} consists of mathematical objects and morphisms (aka
  functions, arrows) between them.  Arrows have a direction and point from
  domain to codomain\footnote{This terminology may be a misnomer. See
    \nameref{concreteandabstract}. Also note that specifying only the domain and
  codomain does not necessarily specify the arrow.}. For every object $\mathfrak
  o$ there is an identity arrow $\Id: \mathfrak o \to \mathfrak o$. Arrows are
  composable with each other associatively, with the caveat that the ``resultant'' arrow is
  already present in the category; composition does not create ``new'' arrows. 
  \[ f\Id_{\mathfrak o} = \Id_{\mathfrak o'}f = f \text{ for all objects
  }\mathfrak o, \mathfrak o' \text{ and }f : \mathfrak o \to \mathfrak o'\]
\end{defn}

What differentiates category theory from set theory is the
\href{https://ncatlab.org/nlab/show/category+theory}{focus} of category theory
on relations between objects\footnote{\cite{riehl2017category} pp. 3 asks to
consider the object ``in tandem''. However, I feel it is more of an obligation
to consider the object, as they appear only to define what morphisms are (cf.
\href{https://ncatlab.org/nlab/show/category+theory}{nLab}). For instance, just
look at the number of axioms morphisms are supposed to satisfy.} rather than
trying to describe ``more complex'' objects in terms of ``more atomic'' objects.

\begin{rmk}[Non-Trivial or Trivial?]
  \label{s1}
  Unlike most other mathematical objects we encounter, a category with one
  object need not be as trivial as its mundane siblings. For example, consider
  $\Z$ with functions $\Z \to \Z$ being the morphsims. An infinite number of
  morphisms are possible even if there is only one object!
\end{rmk}

An \hi{endomorphism} is an arrow from an object to itself. By the last remark,
you know that endomorphisms on even one object can be quite
elaborate. However not all endomorphisms are born equal.

\begin{defn}[Isomorphism]
  We can now describe a purely category theoretic definition of an isomorphism.
  An arrow $f: \mathfrak o \to \mathfrak o'$ is an \emph{isomorphism} if there
  exists an arrow $g: \mathfrak o' \to \mathfrak o$ satisfying the relations,
  \begin{align*}
    fg &= \Id_{\mathfrak o'} \\
    gf &= \Id_{\mathfrak o}.
  \end{align*}
  If an isomorphism exists between $\mathfrak o$ and $\mathfrak o'$, then
  they are \emph{isomorphic}, i.e., $\mathfrak o \cong \mathfrak o'$.
\end{defn}

An endomorphism that is also an \emph{isomorphism} is called an \hi{automorphism}.

\begin{defn}[Groupoid]
  A category where every morphism is an isomorphism is called a groupoid.
  For example, a groupoid with one object is precisely a group (also, every
  element of a group $g \in G$ is associated with a canonical isomorphism
  $x \mapsto gx$; in a particular category though, you may wish to declare a
  subset of these possible isomorphisms as arrows -- there you have a
  \emph{quotient group}).
\end{defn}
\begin{defn}[Concrete and Abstract]
  \label{concreteandabstract}
  Time for a surprise! A category where all arrows are functions on the domain
  (as a set of values) and take values in codomain (as a set of values) is known
  as \emph{concrete}. The category is said to be \emph{abstract} otherwise.
\end{defn}

\begin{defn}[Smallness]
  A category is said to be \emph{locally small} if the collection of morphisms
  between any given pair of objects, forms a set. The category is \emph{small}
  if the collection of all arrows inside it forms a set.
\end{defn}

\begin{defn}[Hom-set]
  The collection of morphisms between a fixed
  pair of objects in a category, say $X \to Y$, is denoted by $\Hom(X, Y)$.
  In a locally small category, this is called the Hom-set (even if it may not
  be a set of homomorphism).
\end{defn}

\section{Functors}
%% Prove that C(c, -) is a functor
%% f: x \to y ==> f_*: \Hom(c, x) \to \Hom(c, y)

\newpage

\part{Problems and Solutions}
\setcounter{chapter}{0}
TO BE ADDED FROM ANOTHER FILE DEDICATED TO PROBLEMS AND THEIR SOLUTIONS.

\part{Crash Courses}
\setcounter{chapter}{0}

\chapter{Group Theory}

\begin{description}
  \item[Normal Subgroup] $G$ is a normal subgroup of $H$, denoted by 
    \[G \triangleleft H \iff hgh^{-1} \in G\] 
    for every $g \in G$ and $h
    \in H$. The map $x \mapsto hxh^{-1}$ is known as a \emph{conjugation}. They
    are kernels of some homomorphism. Quotient groups can only be defined for
    normal subgroups.

  \item[Short exact sequences]  A sequence of groups with homomorphisms
    between them, i.e., 

    \[ \dots \to G_{-1} \to G_{0} \to G_{1} \to G_{2} \to \dots \]

    is said to be \emph{exact} if the image of the map $\psi: G_{k-1} \to G_k$ 
    is the kernel of the map $\phi: G_k \to G_{k+1}$ for each $k$.

  \item[Group Extension] A group extension of an abelian group $H$ by an abelian
    group $G$ consists of a group $E$ so that 
    \[ 0 \to G \to E \to H \to 0 \]
    is a short exact sequence. Note that this embeds $G$ as a normal subgroup of
    $E$ such that $H \approx E/G$. The inclusion $\psi: G \hookrightarrow E$ and 
    surjection $\phi: E \twoheadrightarrow H$ are a part of the group extension.

  \item[Prufer p-group] It is the Sylow $p$-subgroup of $\Q/\Z$, i.e.,
    $\Z[1/p]/\Z$. This is so because all elements having their order as a prime
    power of $p$ satisfy,
    \[p^k \cdot \frac ab \in [0] \implies \frac ab = \frac n{p^k} \text{ for
    some } n \in \Z.\]

\end{description}



\bibliographystyle{alpha}
\bibliography{references.bib}
 
\end{document}
