\documentclass{scrbook}

\usepackage{amsmath}
\usepackage{amssymb}
\usepackage{amsthm}
\usepackage{xcolor}
\usepackage{hyperref}
\usepackage{datetime}

\newdateformat{monthyear}{
  \monthname[\THEMONTH] \THEYEAR
}

\theoremstyle{definition}\newtheorem{defn}{Definition}
\theoremstyle{definition}\newtheorem*{rmk}{Remark}

\newcommand{\Chapter}[1]{\chapter*{#1}\addcontentsline{toc}{chapter}{#1}}
\newcommand{\hi}[1]{{\color{blue}\textbf{#1}}}

\newcommand{\Z}{\mathbf{Z}}
\newcommand{\Q}{\mathbf{Q}}
\newcommand{\R}{\mathbf{R}}

\DeclareMathOperator{\Hom}{Hom}

\author{\textsc{Bhavya Tiwari}\thanks{Roll number \textsc{24B0913}}}
\title{Category Theory \\[5mm]
\Large Summer of Science 2025 Project}

\begin{document}
\maketitle
\tableofcontents

\Chapter{Preface}

This report is a collection of things I have learnt and problems that I have
solved during my Summer of Science 2025 project on \emph{Category Theory} under
Aryaman Maithani.
\\[2mm]
Each chapter begins with a section titled \hi{Terminology Revisited} and ends
with solutions to selected problems from the references. It is not necessary
that the terminology section has to be thoroughly understood before proceeding
to read the chapter; infact it is quite the opposite. If some advanced example
from \emph{Category Theory in Context} took time for me to process, I add some
terms in the corresponding chapter so that it is easier to go through next time.
Anyway we can't keep reading a new book to understand an example in a category
theory book; the program is called Summer of Science and not Sages of Science.
\\[2mm]
More Details to be added later.

\bigskip\noindent
Bhavya Tiwari, IIT Bombay \\
\monthyear\today

\chapter{Introduction to Categories}
\section*{Terminology Revisited}

\begin{description}
  \item[Normal Subgroup] $G$ is a normal subgroup of $H$, denoted by 
    \[G \triangleleft H \iff hgh^{-1} \in G\] 
    for every $g \in G$ and $h
    \in H$. The map $x \mapsto hxh^{-1}$ is known as a \emph{conjugation}. They
    are kernels of some homomorphism. Quotient groups can only be defined for
    normal subgroups.

  \item[Short exact sequences]  A sequence of groups with homomorphisms
    between them, i.e., 

    \[ \dots \to G_{-1} \to G_{0} \to G_{1} \to G_{2} \to \dots \]

    is said to be \emph{exact} if the image of the map $\psi: G_{k-1} \to G_k$ 
    is the kernel of the map $\phi: G_k \to G_{k+1}$ for each $k$.

  \item[Group Extension] A group extension of an abelian group $H$ by an abelian
    group $G$ consists of a group $E$ so that 
    \[ 0 \to G \to E \to H \to 0 \]
    is a short exact sequence. Note that this embeds $G$ as a normal subgroup of
    $E$ such that $H \approx E/G$. The inclusion $\psi: G \hookrightarrow E$ and 
    surjection $\phi: E \twoheadrightarrow H$ are a part of the group extension.

  \item[Prufer p-group] It is the Sylow $p$-subgroup of $\Q/\Z$, i.e.,
    $\Z[1/p]/\Z$. This is so because all elements having their order as a prime
    power of $p$ satisfy,
    \[p^k \cdot \frac ab \in [0] \implies \frac ab = \frac n{p^k} \text{ for
    some } n \in \Z.\]

  \item[Homology and Cohomology groups] They are used in algebraic topology.

  \item[Hom-set] The collection of morphisms between a fixed
    pair of objects in a category, say $X \to Y$, is denoted by $\Hom(X, Y)$.
    In a locally small category, this is called the Hom-set (even if it may not
    be a set of homomorphism).
\end{description}

\newpage

\section{What is a Category?}

\begin{defn}[Category]
  A \emph{category} consists of mathematical objects and morphisms (aka
  functions, arrows) between them.
   Arrows have a direction and point from domain to the codomain. For
  every object there is an identity arrow from the object to itself. Arrows are
  composable with each other, with the caveat that the ``resultant'' arrow is
  already present in the category; composition does not create ``new''
  arrows. Composing a morphism with the identity morphism on either its
  domain or codomain yields the same morphism. Note, however, that you must
  left-compose the identity morphism on the codomain and right-compose the one
  on the domain. As usual, composition should be associative.
\end{defn}

What differentiates category theory from set theory is the
\href{https://ncatlab.org/nlab/show/category+theory}{focus} of category theory
on relations between objects (considered in tandem with the objects
themselves\footnote{As said in \cite{riehl2017category} pp. 3}) rather than
trying to come up with a description of objects in terms of ``more atomic''
objects.

\begin{rmk}[Non-Trivial or Trivial?]
  \label{s1}
  Unlike most other mathematical objects we encounter, a category with one
  object need not be as trivial as those mundane siblings. For example, consider
  $\Z$ with functions $\Z \to \Z$ being the morphsims. An infinite number of
  morphisms are possible even if there is only one object!
\end{rmk}

An \emph{endomorphism} is an arrow from an object to itself. By
\nameref{s1}, you know that endomorphisms on even one object can be quite
elabborate. However not all endomorphisms are born equal. An endomorphism that
is also an isomorphism is called an \emph{automorphism}.

\begin{defn}[Groupoid]
  A category where every morphism is an isomorphism is called a groupoid.
  For example, a groupoid with one object is precisely a group (also, every
  element of a group $g \in G$ is associated with a canonical isomorphism
  $x \mapsto gx$; in a particular category though, you may wish to declare a
  subset of these possible isomorphisms as arrows -- there you have a
  \emph{quotient group}).
\end{defn}

\begin{defn}[Concrete and Abstract]

\end{defn}

\begin{defn}[Smallness]
  A category is said to be \emph{locally small} if the collection of morphisms
  between any given pair of objects, forms a set. The category is \emph{small}
  if the collection of all arrows inside it forms a set.
\end{defn}

\newpage

\section{Solutions to Selected Problems}
\subsection{Category Theory in Context}
\subsection{Tom Leinster}

\bibliographystyle{alpha}
\bibliography{references.bib}
 
\end{document}
