\documentclass{scrbook}

\usepackage{amsmath}
\usepackage{amssymb}
\usepackage{amsfonts}
\usepackage{xcolor}
\usepackage{hyperref}
\usepackage{amsthm}
\usepackage{enumitem}

\newcommand{\hi}[1]{{\color{blue}\textbf{#1}}}

\newcommand{\Z}{\mathbf{Z}}
\newcommand{\Q}{\mathbf{Q}}
\newcommand{\R}{\mathbf{R}}

\DeclareMathOperator{\Hom}{Hom}

\setlength{\parindent}{0pt}

\author{\textsc{Bhavya Tiwari}\thanks{Roll number \textsc{24B0913}}}
\title{Category Theory \\[5mm]
\Large Summer of Science 2025 Project}

\begin{document}
\maketitle
\tableofcontents

\part{Category Theory in Context}
\chapter{Categories, Functors, Natural Transformations}
\section{Abstract and concrete categories}

\hi{1.1.i}
\begin{proof}
  Observe that $gfh : y \to x$. Also $gfh = (gf)h = h$ and
  $gfh = g(fh) = g$, thus $g = h$. This means that $f$ has an
  inverse morphism $g$ such that $fg = 1_x$ and $gf = 1_y$, so
  $f$ must be an isomorphism.
\end{proof}

\hi{1.1.ii}
\begin{proof}
  First we show that the collection of isomorphisms in
  $\mathcal{C}$ with the objects of $\mathcal{C}$ forms a
  category. The identity morphism exists in the new category
  since it is its own inverse and therefore an isomorphism.
  Associativity is inherited from morphisms on
  $\mathcal C$. To prove composability, say $f: x \to y$ and
  $g: y \to z$ are isomorphisms, i.e., there exist $f_*$ and
  $g_*$ such that $ff_* = 1_y$, $f_*f = 1_x$ and $gg_* = 1_z$,
  $g_*g = 1_y$.  Clearly $gf$ is a morphism since so are $f$
  and $g$. We claim that $f_*g_*$ is its inverse morphism,
  which is trivial to verify. Hence $gf$ is an isomorphism.
  It is clear that the resultant category is a groupoid so let
  us prove its maximality. Suppose to the contrary that there
  is a larger groupoid $\mathcal C'$ containing it. Since they have the same
  objects, $\mathcal C'$ must contain an isomorphism not in
  $\mathcal C$. This is a contradiction because $\mathcal C$
  contains all the isomorphisms between its objects.
  \end{proof}

\hi{1.1.iii}
\begin{proof}
Associativity follows in either case because the maps in
$\mathcal C$ are associative.
  \begin{enumerate}[label=(\roman*)]
    \item First note that there is an identity morphism for
      every $f: c \to x$ because we can take $g = f$. To
      show composability, let $h_1: a \to b$ be from $f: c \to a$ to
      $g: c \to b$ and $h_2: b \to d$ be from $g: c \to b$ to $h: c
      \to d$. Note that $g = h_1f$ and $h = h_2 g$ implies
      $h = h_2(h_1f) = (h_2h_1)f$ and so the diagram with
      $h_2h_1$ also commutes.

    \item Identity morphism for $f$ is found by taking $g =
      f$. To show composability, proceed as before $f = gh_1
      = (hh_2)h_1 = h(h_2h_1)$.
  \end{enumerate}  
\end{proof}

\section{Duality}

\part{Tom Leinster}

\end{document}
